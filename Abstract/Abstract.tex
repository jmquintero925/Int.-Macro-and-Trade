\documentclass[12pt,oneside,reqno]{amsart}
\usepackage[utf8]{inputenc}
\usepackage[left=3cm,right=3cm,top=2.5cm,bottom=2.5cm]{geometry}
\usepackage{amsmath,amsfonts,amssymb,xfrac,enumitem,xcolor}
\usepackage{fancyhdr}
\pagestyle{fancy}

\newcommand{\pr}{\mathbb{P}\mathrm{r}}

\title{Skilled immigration, firms, and policy}
\author{Mishita Mehra \and Hewei Shen}
\date{December 2022. Referee report presented by Jose M. Quintero for International Macroeconomics and Trade.}

\lhead{Jose M. Quintero}
\rhead{Int. Macro and Trade.} 

\begin{document}

\maketitle 

\section{Original Abstract}
We develop a two-sector dynamic general equilibrium model with skilled immigration that focuses on the role of firms. The model is consistent with data on firm demand for foreign labor and firm size in the United States. Monopolistically competitive firms in the skill-intensive sector differ in productivity and a subset of relatively larger firms hire foreign workers subject to skilled immigration policies that are similar to the United States: Firms face hiring costs and there is an immigration cap that binds if the aggregate demand for foreign workers exceeds a quota. A binding immigration cap reduces firms’ demand for foreign skilled workers and, quantitatively, accounts for 16.82\% of the hiring distortion in the decentralized equilibrium when compared to the first-best allocation. We evaluate the effects of different immigration cap changes and quantify the welfare implications for domestic households.

\section{My Abstract}
Skilled immigration to the U.S. has increased by 58\% in recent years. Firms play a crucial role in skilled migration as they need to sponsor visas. Yet firms' demand is constrained by a policy-imposed cap. In this paper, we explore the effects of limiting the flow of skilled immigrants. To do so, we build a DGE model with multiple sectors and heterogeneous firms that decide how many domestic and foreign workers to hire. The model estimates that the policy accounts for 16.82\% of the hiring distortion. Finally, the effects on domestic workers are heterogeneous

\end{document}
