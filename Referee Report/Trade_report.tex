\documentclass[12pt,oneside,reqno]{amsart}
\usepackage[utf8]{inputenc}
\usepackage[left=3cm,right=3cm,top=2.5cm,bottom=2.5cm]{geometry}
\usepackage{amsmath,amsfonts,amssymb,xfrac,enumitem,xcolor}
\usepackage{fancyhdr}
\pagestyle{fancy}

\newcommand{\pr}{\mathbb{P}\mathrm{r}}

\title{Trade and Informality in the Presence of Labor Market Frictions and Regulations}
\author{Rafael Dix-Carneiro \and Pinelopi K. Goldberg \and Costas Meghir \and Grabiel Ulyssea}
\date{December 2022. Referee report presented by Jose M. Quintero for}

\lhead{Jose M. Quintero}
\rhead{Int. Macro and Trade.} 

\begin{document}

\maketitle 

In this paper, the authors try to fill a gap in the trade literature by building a general equilibrium model with an informal sector in the labor market. The main argument for doing this is that informality is a salient characteristic of developing economies, and the interaction between trade policies with informal labor markets has yet to be explored through the lens of a GE model. Specifically, the authors argue that all research that has accounted for informality upon a change in terms of trade leverages regional variation. Thus, by building a GE model, this paper can perform several counterfactual exercises and speak to the welfare effects of trade under the presence of informality. 

By adding an informal sector, the model builds on \textcolor{red}{Cosar et al. (2016)}. Here is a quick summary of the model:

\begin{enumerate}[label=\arabic*.]
    \item First, a representative agent consumes a bundle of consumption goods and services. The difference between consumption goods and services is that services are not tradable and, thus, will not face foreign competition upon a change in trade costs.
    \item A monopolistic firm produces each intermediate input. Firms are heterogeneous in their TFP as in \textcolor{red}{Melitz (2003)}, which follows a log-normal AR(1) process. Firms produce using labor, consumption goods, and services as inputs.
    \item Within each sector, firms can opt to be either formal or informal. If a firm selects itself into the formal sector, it will be subject to labor market regulations such as taxes, firing costs, minimum wages, and unemployment benefits.
    \item On the other hand, informal firms can avoid labor market regulations but are subject to a convex cost function that increases with size. This captures all the costly operations for informal firms, such as avoiding monitoring from government officials. Moreover, informal firms that operate in the tradable sector can not export their production.
    \item The trade-off between informality and formality has two dimensions: first, the marginal cost of labor is constant and initially higher in the formal sector, but as firms expand, the marginal cost of informality will become larger given the convexity assumption. This will yield a unique threshold for productivity that will determine the decision between formality and informality. The second dimension is only for firms in the tradable sector, where firms cannot trade their production, and thus the market access will be lower even though they are set to face tougher competition.
    \item  To expand, firms need to post costly vacancies. The cost of posting a vacancy increase with the percentage increase of the labor, and there is random matching between workers and firms, which allows for unemployment. 
    \item A Nash bargaining process between workers and firms determines wages in each sector. 
    \item Finally, there are iceberg costs and tariffs whenever firms want to export/import. 
\end{enumerate}
The authors then calibrate the model using several data sources from Brazil. Once the model is calibrated, the authors do counterfactual exercises by changing the iceberg costs of trade and comparing the aggregate the model yields on the steady-state equilibrium. 

This paper seems very relevant to the literature as many trade liberalizations studies have neglected to consider this margin of adjustment. The sole nature of informality makes it hard to quantify the gains of trade in such economies. Hence, building a GE model to estimate the welfare effects of trade in the presence of informality seems like a natural step. The model is simple enough to highlight the mechanisms the authors have in mind. Also, Brazil's rich data sources make for a perfect setting to perform the analysis as it is an economy with a high informality rate and a trade liberalization episode during the 1990s. 

On the more negative side, the paper has a few shortcomings that must be addressed before this becomes a publishable article. My main concern is an assumption embedded in the model that might drive some results mechanically. The functional form for the cost of posting a job for a firm of size $\ell$ is 
\begin{equation*}
    H(\ell,\ell') = \left(\mu_{kj}^v\right)^{-\gamma_{k_1}}\left(\frac{h_k}{\gamma_{k_1}}\right)\left(\frac{\ell'-\ell}{\ell^{\gamma_{k_2}}}\right)^{\gamma_{k_1}}
\end{equation*}
for both formal and informal firms. Note that the cost of finding a job is decreasing in firm size. This functional form intends to capture that firm growth rate decreases with size, as shown in one of the paper's stylized facts in section 4. The problem is that the data sample used to calculate the growth by firm size only considers formal firms and do not speaks about informal firms. Moreover, informal firms are mechanically meant to be smaller due to the convex cost. Consequently, by construction, the model makes posting jobs expensive for informal firms relative to formal ones. This does not seem consistent with anecdotal evidence nor with previous results found in the literature (See \textcolor{red}{Costas et al. (2016)}), where the job-finding rate in informality is higher. One source of relief could be that the authors are hitting the transition rates (See Table A2 in appendix K). However, upon further inspection, I noticed that $\phi$ is a free parameter used to match the transition rates (see footnote 23). Regardless, I wonder how much of the results are being driven by the fact that informal firms are constrained in their size and their capacity to adjust their size. This becomes particularly relevant as it will decrease the value function of the informal firm and thus impact the informal sector's bargaining problem and wage setting. Other equilibrium objects, such as entry rate to the informal sector, will change. Furthermore, given that the counterfactuals are speaking to changes in wage inequality, informality share,  and welfare, it seems like the first order of business to address the plausibility of this assumption. 

The remaining of my concerns are second order. First, there is no discussion about the unicity of the equilibrium. It is not clear to me that when adding informality, bargaining for wage setting, and a bunch of regulations (including minimum wage), the unique equilibrium result from \textcolor{red}{Melitzs (2003)} still holds. This will be particularly relevant for the counterfactual exercises. During the counterfactual exercises, the model compares the economy's steady-state equilibrium for different values of iceberg costs. However, if there were to be multiple equilibria, then the question of ``to which equilibrium is the economy transitioning?'' becomes pertinent. This does not seem to be a central point of the paper. Along the same lines, since the model only cares about steady-state equilibrium and there is nothing about a firm life cycle, I wonder why the authors assume that productivity follows an AR(1) process. Especially when the calibrated values imply that the persistence is very close to one and the variance of the normal shock is low, perhaps one line justifying the necessity of this assumption would do. Finally, the empirical facts, although only used to motivate, can be presented more cleanly. Facts 3 and 4 mixes two different data sources. Fact 1 can show the share of informal firms by firm size (given that it is only considering firms with five or fewer employees) and avid the question of which source of variation the linear regression is exploiting. 


\end{document}
