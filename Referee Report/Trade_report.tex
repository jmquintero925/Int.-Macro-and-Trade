\documentclass[12pt,oneside,reqno]{amsart}
\usepackage[utf8]{inputenc}
\usepackage[left=3cm,right=3cm,top=2.5cm,bottom=2.5cm]{geometry}
\usepackage{amsmath,amsfonts,amssymb,xfrac,enumitem,xcolor}
\usepackage{fancyhdr}
\pagestyle{fancy}

\newcommand{\pr}{\mathbb{P}\mathrm{r}}

\title{Trade and Informality in the Presence of Labor Market Frictions and Regulations}
\author{Rafael Dix-Carneiro \and Pinelopi K. Goldberg \and Costas Meghir \and Grabiel Ulyssea}
\date{December 2022. Referee report presented by Jose M. Quintero for}

\lhead{Jose M. Quintero}
\rhead{Int. Macro and Trade.} 

\begin{document}

\maketitle 

\begin{enumerate}[leftmargin=0pt, label=\textbf{\arabic*.}]
    \item The Paper
    \begin{enumerate}[label=\textbf{\alph*)}]
        \item What are the effects of trade liberalization in developing economies?
        \item Informality being a salient trait of developing economies needs to be considered when studying trade liberalization. 
        \item Takes Cosar (2016) as a frame of reference, the model develops a GE equilibrium model (See my notes on the model) 
        \item Merges several data sources from Brazil. 
        \item Counterfactuals: how are aggregate variables responding to costs of trade? 
    \end{enumerate}
    \item The importance of the idea
    \begin{enumerate}[label=\textbf{\alph*)}]
        \item Very relevant to consider how is informality interacting with trade forces. 
        \item The sole nature of informality requires a GE model to perform the analysis. 
        \item The availability of the data makes Brazil a perfect setting. 
    \end{enumerate}
    \item Shortcomings
    \begin{enumerate}[label=\textbf{\alph*)}]
        \item Big assumption: The cost of posting is higher mechanically for informal firms. 
        \item This assumption is inconsistent with anecdotal and empirical evidence. There is no justification for why this assumption is posted, but it concerns me that it might be driving many of the results. 
        \begin{enumerate}[label=\textbf{\roman*.}]
            \item It affects the bargaining problem between informal firms and workers, changing the wage of the informal sector. 
            \item Implications over inequality, entry, and overall the aggregate variables of interest. 
        \end{enumerate}
        \item Not a discussion about the unicity of the equilibrium. This is important as all the counterfactual exercises compare equilibria for different parameters. In a situation where multiple equilibria arise, the transitional dynamics become very important. 
        \item There is a slippage between the motivation and the counterfactual exercises. More relevant to study the effects of tariffs as this is a policy instrument. 
    \end{enumerate}
    \item Minor recommendations. 
    \begin{enumerate}[label=\textbf{\alph*)}]
        \item Improvement to how some of the empirical facts are presented. 
        \begin{enumerate}[label=\textbf{\roman*.}]
            \item Fact 2: Is there enough variation with only five bins?
            \item Fact 3: It's weird to compare the regressions from different samples. There is a sample selection. 
            \item Fact 4: Mix matching again data sources opens up more questions. Formal firms are subject to regulations even while being small. The implied wage of formal workers should be higher even among firms with fewer than five employees. 
        \end{enumerate}
        \item Why does the model requires that firm-specific productivity follows an AR(1)? It seems like an extra complication that adds two parameters to the estimation. 
        
    \end{enumerate}
\end{enumerate}



\end{document}
